% mnras_template.tex
%
% LaTeX template for creating an MNRAS paper
%
% v3.0 released 14 May 2015
% (version numbers match those of mnras.cls)
%
% Copyright (C) Royal Astronomical Society 2015
% Authors:
% Keith T. Smith (Royal Astronomical Society)

% Change log
%
% v3.0 May 2015
%    Renamed to match the new package name
%    Version number matches mnras.cls
%    A few minor tweaks to wording
% v1.0 September 2013
%    Beta testing only - never publicly released
%    First version: a simple (ish) template for creating an MNRAS paper

%%%%%%%%%%%%%%%%%%%%%%%%%%%%%%%%%%%%%%%%%%%%%%%%%%
% Basic setup. Most papers should leave these options alone.
\documentclass[fleqn,usenatbib]{mnras}

% MNRAS is set in Times font. If you don't have this installed (most LaTeX
% installations will be fine) or prefer the old Computer Modern fonts, comment
% out the following line
%\usepackage{newtxtext,newtxmath}
% Depending on your LaTeX fonts installation, you might get better results with one of these:
%\usepackage{mathptmx}
%\usepackage{txfonts}

% Use vector fonts, so it zooms properly in on-screen viewing software
% Don't change these lines unless you know what you are doing
\usepackage[T1]{fontenc}
\usepackage{ae,aecompl}

%%%%% AUTHORS - PLACE YOUR OWN PACKAGES HERE %%%%%

% Only include extra packages if you really need them. Common packages are:
\usepackage{graphicx}	% Including figure files
\usepackage{amsmath}	% Advanced maths commands
\usepackage{amssymb}	% Extra maths symbols

%%%%%%%%%%%%%%%%%%%%%%%%%%%%%%%%%%%%%%%%%%%%%%%%%%

%%%%% AUTHORS - PLACE YOUR OWN COMMANDS HERE %%%%%

% Please keep new commands to a minimum, and use \newcommand not \def to avoid
% overwriting existing commands. Example:
%\newcommand{\pcm}{\,cm$^{-2}$}	% per cm-squared

%%%%%%%%%%%%%%%%%%%%%%%%%%%%%%%%%%%%%%%%%%%%%%%%%%
\newcommand{\Msun}{\,{\rm M}$_{\odot}$\,}
\newcommand{\Msunh}{\,{\rm M}$_{\odot}$\,\ifmmode h^{-1}\else $h^{-1}$\fi}
\newcommand{\Mpch}{\,{\rm Mpc}\,\ifmmode h^{-1}\else $h^{-1}$\fi}
\newcommand{\kpch}{\,{\rm kpc}\,\ifmmode h^{-1}\else $h^{-1}$\fi}
\newcommand{\kpc}{\,{\rm kpc}\,}
%%%%%%%%%%%%%%%%%%% TITLE PAGE %%%%%%%%%%%%%%%%%%%

% Title of the paper, and the short title which is used in the headers.
% Keep the title short and informative.
\title[Galaxy Assembly Bias]{The Assembly Time Dependence of Galaxy Clustering}

% The list of authors, and the short list which is used in the headers.
% If you need two or more lines of authors, add an extra line using \newauthor
\author[Camargo, Y. \& Forero-Romero J. E.]{
Y. Camargo,$^{1}$\thanks{E-mail: mn@ras.org.uk}
J. E. Forero-Romero,$^{2}$\thanks{E-mail: je.forero@uniandes.edu.co}
\\
% List of institutions
$^{1}$Universidad Nacional de Colombia, Bogot\'a, Colombia\\
$^{2}$Departamento de F\'isica, Universidad de los Andes, Cra. 1 No.
18A-10, Edificio Ip, Bogot\'a, Colombia\\
}

% These dates will be filled out by the publisher
\date{Accepted XXX. Received YYY; in original form ZZZ}

% Enter the current year, for the copyright statements etc.
\pubyear{2015}

% Don't change these lines
\begin{document}
\label{firstpage}
\pagerange{\pageref{firstpage}--\pageref{lastpage}}
\maketitle

% Abstract of the paper
\begin{abstract}
    We use a high resolution cosmological hydro-dynamical simulation
    to investigate stellar mass assembly 
    and its impact on clustering at $z=0$.
    Using the merger trees for galaxies in the range $10^{9}$\Msunh $\leq M_{\star} \leq 10^{12.5}$ \Msunh we first find a transition mass of $10^{10.5}$\Msunh that divides two
    correlation regimes between assembly time and stellar mass.
    Above that mass, late assembly times correlate with more massive
    galaxies; below that mass late assembly correlates with less massive
galaxies.
We then quantify how, at fixed stellar mass, galaxies with an early
assembly
are more clustered than late assembly galaxies, regardless if the galaxy
is a central or a satellite.
Finally, for stellar masses below $10^{10}$ \Msunh we find 
cuts in the star-formation rate and the $(g-r)$ colour that mimic
the bias found for early/late assemblying galaxies.
These results can be used to.
Furthermore, these findings give further support to the idea that at
$z=0$  massive galaxies above $10^{10.5}$\Msunh do not show a strong
stellar mass assembly bias.
\end{abstract}

% Select between one and six entries from the list of approved keywords.
% Don't make up new ones.
\begin{keywords}
keyword1 -- keyword2 -- keyword3
\end{keywords}

%%%%%%%%%%%%%%%%%%%%%%%%%%%%%%%%%%%%%%%%%%%%%%%%%%

%%%%%%%%%%%%%%%%% BODY OF PAPER %%%%%%%%%%%%%%%%%%

\section{Introduction}
In the Standard Cosmological Model, Lambda Cold Dark Matter
($\Lambda$CDM) galaxies evolve inside dark matter (DM) halos.
DM halos form through gravitational and grow through successive
mergers.
The physical properties of the galaxies are in turn determined by
cooling and condensation of gas within of halos and mergers with
another galaxies.

The clustering properties of halos are mainly dependent on their mass.
i.e., more massive halos are more clustered than
less massive halos \citep{1984ApJ...284L...9K}.
\cite{2005MNRAS.363L..66G} used N-body simulations to show that at
fixed halo mass halos that assembly early are clustered more strongly
than halos with a late assembly.
They also showed that this effect is stronger for lower masses.
This effect is known as assembly halo bias.

In terms the galaxy-halo relationship the Halo Occupation Distribution
model (HOD) is used to explain the observed effects of different types
of galaxies towards the conditional distribution function
$P(N_G|M_H)$, this statistical connection ignores the possibility that
galaxy properties may be correlated with halo properties different to
halo mass, i.e., this model assumes that halo mass alone suffices to
determine the physical properties of halo inner galaxy population,
however, recent studies have been shown additional dependencies to
time halo formation such as spin, shape, velocity dispersion and
concentration, commonly referred as secondary halo assembly bias, in
terms of galaxy-halo relationship, the dependence on galaxy clustering
on the secondary assembly halo bias is yet subject of discussion, to
this relationship usually refereed as galaxy assembly bias. 

From the observational point of view, \citet{Lacerna_2014} find in
SSDS a fixed halo mass $\approx 10^{12}$\Msunh central
galaxies show a weak but significant dependence of clustering
amplitude with the age, i.e, old central galaxies have a higher
clustering that young central galaxies also found in its mock catalog,
similarly \citet{2016PhRvL.116d1301M} show present significant
evidence of halo assembly bias for SDSS based on weak lensing signals
to two samples of very similar halo mass of $M_{200}\approx 1.9\times
10^{14}h^{-1}M_\odot$, 


On the other hand \citet{2016ApJ...819..119L} do not find convincing
evidence for formation bias. They use a sample with an DM
halo mass of $\approx 10^{12}$\Msunh estimated via lensing. The
classification into early and late forming was from from the inferred
SFH of the central galaxies.

\citet{2019MNRAS.485.1196Z} found that is not possible to rule out
galaxy assembly bias. Their methodology is based in an HOD analysis
that populates a DM only simulation in order to match clustering
observations from SDSS DR7. The HOD model that includes galaxy
assembly bias improves the fits to the observational data.

\cite{2020MNRAS.tmp.1844M} used the IllustrisTNG300 simulation to show
that central galaxies present a different clustering signal when split
into DM halos with different formation times. 
A similar signal is present when the central galaxies are split by
stellar mass, colour, specific starf formatio rate, surface density
and galaxy size. 
They report the changes in the galaxy bias as a function of the DM
halo mass.




In the context of the galaxy quenching the
models predict a strong correlation between a fraction quenched and
large scale density environment, using central galaxies $M_{\star}
\gtrsim 10^{10}M_\odot h^{-2}$ in SDSS \citet{2017MNRAS.472.2504T}
show that the halo formation history has a small but statistically
significant impact on quenching of star formation at high masses,
while the process in low-mass central galaxies is uncorrelated with
halo formation history, \citet{2016MNRAS.457.4360Z} using clustering
and galaxy-galaxy lensing of red and blue galaxies in SDSS find that
the halo quenching model provides more fitting to the galaxies above
$10^{11}M_\odot h^{-2}$, predicting the average halo mass of red and
blue central galaxies although it has also been shown that quenching
process has a limited correlation with halo formation history
\citep{2018MNRAS.478.4487T}. 
Using hydrodynamics simulations and
N-body simulations the galaxy assembly bias effect is more visible, in
terms of variations galaxy occupancy of the dark matter halos, halos
at low mass ($\leq 10^{10}$\Msunh) in most dense environment and early
formed halos expected host a central galaxy and fewer satellite
galaxies that those in the least dense environments and late-formed
halos \citep{2018MNRAS.480.3978A} in the galactic conformity effect
galaxies with stellar mass $M_\star > 2\times 10^9 M_\odot$ shows a
significant signal out to 10 Mpc \citep{2016MNRAS.455..185B}, moreover
in Illustris simulations has been demonstrated a stronger correlation
between the central galaxies and the peak maximum circular velocity of
their hosting haloes
\citep{2018arXiv181211210X}\citep{2005ApJ...630....1Z}. 
Also, Semi Analytic models (SAM) show contradictory results, some
studies find that the clustering of galaxies depend not only halo mass
but also on the secondary halo properties, halo central galaxies are
differently affected by assembly bias than are galaxies of all types
(see, e.g., \citep{2019MNRAS.486..582P};
\citep{2007MNRAS.374.1303C};\citep{2019MNRAS.484.1133C};
\citep{2014ApJ...794...74J}), while authors as
\citet{2014MNRAS.443.3044Z} conclude that the galaxy-halo relationship
inferred from galaxy clustering is subject to significant systematic
errors induced by assembly bias. If the galaxy properties depend on
the secondary properties of its parent halo it remains unclear
especially on the observational point of view.   


As previously we mentioned the investigations about the age-clustering
dependence have mainly focused on halo population detecting this
effect (assembly halo bias), i.e., oldest halos are more clustered
than younger haloes, however, the galaxies are the tracers of
formation structure and are used to constrain the cosmological
parameters. Some authors focus their studies on age-clustering
relation in galactic populations, e.g., \citet{2012A&A...539A..46A}
study the properties of galaxy pairs in high-density environments in
SSDS-DR7 detecting young groups and young clusters are associated with
low-density global environments, \citet{2007MNRAS.378..777R} analyzing
the relation between age and clustering within halo catalogs found
that galaxies of a given velocity circular will have formed earlier if
they lie in groups or clusters today, the other hand, using
semi-analytic galaxy formation models and observational data at fixed
stellar mass \citet{2013MNRAS.433..515W} predict that the clustering
of central galaxies depends on the specific star formation rate, found
more passive galaxies have a higher clustering amplitude, but
\citet{Lacerna_2014} also find a weak but significant dependence of
clustering amplitude whit the age of central galaxies in SDDS, at
fixed halo mass they found that old galaxies show a higher clustering
relative the young population and \citet{Berlind:2006eb} found a
connection between halo age and central galaxy, halos that assemble
earlier likely contain redder central galaxies than recently assembled
halos of the same mass. The galaxy assembly bias study in this paper
is therefore the relation between the galaxy age dependence and galaxy
clustering. In this letter, we study the dependence of galaxy
clustering on galaxy assembly for central and satellite galaxies using
the cosmological hydro-dynamical simulation Illustris TNG300-1. In
this way, the definition for galaxy assembly bias indicates the
dependence of galaxy clustering on a secondary galactic property as
the assembly time. 

The outline of this letter is as follows, in section \ref{sec:simul}
we describe the IllustrisTNG300-1 simulations. In section
\ref{sec:galactic_prop} we detail the construction of galaxy sample in
IllustrisTNG300-1, the stellar mass range, the assembly times and the
definition of our age dependence of clustering for central and
satellite galaxies through the relative bias. Finally, the results and
conclusions are discussed in section \ref{sec:conclu}. 


\section{Simulated Galaxy Sample}
\label{sec:simul} % used for referring to this section from elsewhere

In this Letter we use public data from the IllustrisTNG project
\url{https://www.tng-project.org/}. 
We use the results from the simulation labeled as TNG300-1.
This simulation was performed with the AREPO code
\citep{2018MNRAS.473.4077P} that solves gravitaional and
magnetohydrodynamical physics.
The simulation also implements sub-resolution physics to describe the
processes related to star formation, black hole growth and their
associated feedback processes.
The used cosmological parameters are $\Omega_m=0.38089$,
$\Omega_b=0.0486$, $\Omega_\Lambda= 0.6911$ and $h=0.6774$ consistent
with Planck2015 \citep{2016A&A...594A..13P}. 

TNG300-1 is a simulation performed on a cubic volume
with  $205$ \Mpch on a side.
It follows the evolution of $2500^3$ DM computational particles of
mass $3.98 \times 10^7$ \Msunh. 
\textbf{Faltan los datos que cuantifican la resolucion en la
  descripcion del gas y de las estrellas.}



\begin{figure}
    \centering
    \includegraphics[width=1\columnwidth]{figuras/Histogramas.pdf}
    \caption{Stellar mass functions for all galaxies and the partition
      into centrals and satellites. We focus our study in the stellar
      mass range presented in this plot.} 
    \label{fig:stellar_fuction}
\end{figure}


\begin{figure}
    \centering
     \includegraphics[width=1\columnwidth]{figuras/CH.pdf}
    \includegraphics[width=0.9\columnwidth]{figuras/SH.pdf}
    \caption{Relationship between stellar mass and the parent dark
      matter halo mass.} 
    \label{fig:stellar_to_halo}
\end{figure}


\begin{figure}
    \centering
    \includegraphics[width=1\columnwidth]{figuras/median_assembly.pdf}
    \caption{Median redshift of assembly as a function of stellar mass.
    Different symbols correspond to satellites, centrals or all galaxies.
    The stellar mass around $10^{10.5}$\Msunh shows a transition between two 
    regimes of \emph{upsizing} and \emph{downsizing}.}
    \label{fig:median_assembly}
\end{figure}


In the simulation the DM halos were identified using the
Friends-of-Friends (FoF) algorithm with a linking length of 0.2 times
mean interparticle separation. 
Subhalos were identified using SUBFIND algorithm
\citep{2015MNAS.449...49R}. 
We use the baryonic merger trees to estimate the assembly time for the
galaaxies.
These trees were constructed using the SUBLINK algorithm at the
subhalo level \citep{2015MNRAS.449...49R}.

We extract the main branch of the tree by following back in time the
most massive progenitor.
We use this main branch to define the assembly time as the redshift
$z_{for}$ when the galaxy in the branch has exactly half stellar mass
of its final stellar mass.  

We take as the stellar mass the measurement within
twice the stellar half mass radius of each subhalo
\citep{2018MNRAS.475..676S}.
We only use galaxies with stellar masses (at $z=0$) larger than
$10^{9}$\Msunh,   with this selection we end up with 128258 centrals
and  satellites.
This allows us to have galaxies with a well resolved formation
history. 
\textbf{Cuantas galaxias mas masivas que $10^9$ no tienen un merger
  tree?}  
\textbf{Cuantas particulas estelares aproximadamente tienen una
  galaxia de $10^9$ ?} 


Figure \ref{fig:stellar_fuction} shows the stellar mass functions for
this sample.
There we observe that the most massive satellite galaxy has a
mass of $10^{12}$\Msunh.  
We can also see that there are approximately $20$ central galaxies
with a similar high mass. 
In the results we are going to present next we compare centrals and
satellites. 
We impose a  minimum of $50$ galaxies in a mass bin to extract statistics. 
For this reason our results will have an upper mass limit of
$10^{11.5}$\Msunh, even though there are galaxies in the simulation
more massive than this value.

Figure \ref{fig:stellar_to_halo} shows the stellar mass as a function
of the DM halo mass of the host.
The two panels split the galaxies intro centrals and satellites. 
This shows us that centrals follow a relatively tight relationship
between stellar mass and DM halo mass.
On the other hand, satallite galaxies do not follow a clear trend.
For instance, satellite galaxies with stellar masses $\approx
10^{9}$\Msunh are hosted by halos with masses that span almost three
orders of magnitude.

Figure \ref{fig:median_assembly} summarizes the trends for the
assembly times as a function of mass for central and satellite
galaxies.
We first see that central galaxies have a systematic latter assembly
compared to satellite galaxies.
Apart from the normalization, the trends are similar for centrals and
satellites.
The treds is marked by a "transitional stellar mass" around 
$M_\star
10^{10.5}$ \Msunh.
Above that transition massive galaxies assembly late, below that
treshold more massive galaxies assembly early.

In what follows we dissect the assembly time distribution into
early/late assembly and quantify to what extend those differences are
translated into clustering.
We define early assembly as a galaxy with a formation redshift in the
first quartile of assembly time distributions; late
assembly correspond to the fourth quartile.
The formation time distribution is computed in fixed mass bins and
separately for centrals, satellites and the whole galaxy population. 


\section{Clustering dependence on assembly time}


\begin{figure*}
    \centering
    \includegraphics[width=2.0\columnwidth]{figuras/bias_galaxies.pdf}
    \caption{Relative galaxy bias as a function of stellar mass.
    We consider separately all galaxies, satellites and centrals.
    The bias is higher for early assembling galaxies.}
    \label{fig:galaxy_bias}
\end{figure*}

\begin{figure*}
    \centering
    \includegraphics[width=1.8\columnwidth]{figuras/scatter_assembly.pdf}
    \caption{Comparison of the spatial distribution of \emph{early}
      and \emph{late} assembling galaxies. 
    The galaxies included in the plot have stellar masses around
    $10^{10}$\Msunh.  
    Each panel projects all the galaxy positions over the $x$-$y$
    plane.  
    The split between \emph{late} and \emph{early} corresponds to the
    first and fourth quartile in the redshift assembly distribution
    for all galaxies with the same mass. 
    Each panel has $\sim7$k galaxies. }
    \label{fig:comparison}
\end{figure*}




To quantify the changes in galaxy clustering we estimate the relative
galaxy bias by taking the ratio of the 2-point correlation function of two
samples: 
%
\begin{equation}
b_r(r, S|A)= \frac{\xi_S(r)}{\xi_A(r)}, 
\label{eq:relative}
\end{equation}
%
where $\xi_A$ corresponds to the correlation function from a general
sample $A$ that includes all the galaxies in a fixed stellar mass bin,
while $\xi_S$ is the correlation function computed for a sub-sample,
$S$, of galaxies in $A$. 
In our case the sub-samples are built from either the first and last
quartile in the redshift assembly distribution. 
We compute the correlation function in the separation range of
$5$\Mpch\ to $10$\Mpch\ split into $10$ linearly spaced bins. 
The relative bias is computed from the median over those separation bins.
We associate an error bar to the relative bias from the standard
deviation over the same separation bins.


Figure \ref{fig:galaxy_bias} summarizes the main quantitative results
of our Letter.
There we present the relative galaxy bias as a
function of stellar mass for central, satellites, and all galaxies
separately. 
The relative bias for early assembly galaxies is higher in comparison
with the later assembly galaxies.
The clustering difference between early and late assemblying galaxies
is more pronounced towards lower masses.

Both for satellites and centrals, for masses around
$10^{9}$\Msunh\ the amplitude of the correlation function for
late/early assemblying galaxies change by a factor of two with respect
to the reference galaxy sample. 
However, as the mass increases towards $10^{10.5}$ \Msunh\ the relative
bias goes to unity.  
Above this transitional mass the error bars become so large that 
it is not possible to conclude whether the relative bias is different
between early/late assembling galaxies.  

Figure \ref{fig:comparison} provides a qualitative summary of the main
result in this Letter.
The points correspond to galaxies (centrals and satellites combined) 
within the same mass bin around $10^{10}$\Msunh. 
These galaxies are split into early (younger) and late (older)
assembly. 
Galaxies with early assembly tend to be clustered more tightly
clustered than its late-assembly counterparts.

Following Figure \ref{fig:median_assembly} one can see that most of the early
assembling galaxies should be satellites while most of the late
assembling galaxies are centrals. 
One could think that the different clustering pattern could be
explained given that one is taking a look at centrals versus
satellites. 
However, in what follows we show that the assembly/clustering trend also
holds separately for satellites and centrals.  



\section{Discussion and conclusions}
\label{sec:conclu}

In this letter we used an state-of-the-art hydro-dynamical simulation
to study how the clustering of galaxies depends on their assembly
epoch.
We performed the analysis in the mass range $10^{9}$\Msunh $\leq
M_{\star} \leq 10^{11.5}$ for the whole galaxy population and also for 
central and satellite galaxies separately.

We have showna clustering-age  dependence for central and
satellite galaxies similar to found by \citet{2005MNRAS.363L..66G}
for DM halos. 
At fixed stellar mass early-assembly galaxies tend to be more
clustered than late-assembly galaxies.
This trend is stronger towards lower stellar masses below a mass
treshold of $10^{10.5}$
\Msunh. 
Above this threshold the data does not allow to affirm whether there
is a differet clustering between early and late assembly. 


Using the two-point correlation function, we quantify the clustering
towards the relative bias defined as the equation
\ref{eq:relative}. In fig \ref{fig:galaxy_bias}, we present the
relative bias as a function of stellar mass for all, satellites, and
central galaxies separately. At fixed stellar mass older galaxies to
low stellar mass tend to be more tightly clustered than its
late-assembly counterpart according to e.g., \citep{Lacerna_2014},
\citep{2013MNRAS.433..515W}, \citep{2009MNRAS.394.2229Z}. We find that
in all cases the relative bias is higher for older galaxies and 


To conclude worth noting, that the central and satellite galaxies
evolve in different ways, however, the relationship age-clustering is
the same, i.e, older galaxies prefer overdense regions than younger
galaxies so that the galaxy properties should depend significantly on
the assembly history of their haloes, the old haloes follow the
large-scale cosmic web quite closely, while the distribution of young
haloes looks almost uniform. 



* The effect weakens with increaing mass. Following 

* A prediction of our work is that performing the analysis on
satellite galaxies of groups should also yield a detection.

\section*{Acknowledgements}

a gustavo por prestarnos el aparato jijij


\bibliographystyle{mnras}
\bibliography{bibliography.bib}


% Don't change these lines
\bsp	% typesetting comment
\label{lastpage}
\end{document}
