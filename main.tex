% mnras_template.tex
%
% LaTeX template for creating an MNRAS paper
%
% v3.0 released 14 May 2015
% (version numbers match those of mnras.cls)
%
% Copyright (C) Royal Astronomical Society 2015
% Authors:
% Keith T. Smith (Royal Astronomical Society)

% Change log
%
% v3.0 May 2015
%    Renamed to match the new package name
%    Version number matches mnras.cls
%    A few minor tweaks to wording
% v1.0 September 2013
%    Beta testing only - never publicly released
%    First version: a simple (ish) template for creating an MNRAS paper

%%%%%%%%%%%%%%%%%%%%%%%%%%%%%%%%%%%%%%%%%%%%%%%%%%
% Basic setup. Most papers should leave these options alone.
\documentclass[a4paper,fleqn,usenatbib]{mnras}

% MNRAS is set in Times font. If you don't have this installed (most LaTeX
% installations will be fine) or prefer the old Computer Modern fonts, comment
% out the following line
\usepackage{newtxtext,newtxmath}
% Depending on your LaTeX fonts installation, you might get better results with one of these:
%\usepackage{mathptmx}
%\usepackage{txfonts}

% Use vector fonts, so it zooms properly in on-screen viewing software
% Don't change these lines unless you know what you are doing
\usepackage[T1]{fontenc}
\usepackage{ae,aecompl}

%%%%% AUTHORS - PLACE YOUR OWN PACKAGES HERE %%%%%

% Only include extra packages if you really need them. Common packages are:
\usepackage{graphicx}	% Including figure files
\usepackage{amsmath}	% Advanced maths commands
\usepackage{amssymb}	% Extra maths symbols

%%%%%%%%%%%%%%%%%%%%%%%%%%%%%%%%%%%%%%%%%%%%%%%%%%

%%%%% AUTHORS - PLACE YOUR OWN COMMANDS HERE %%%%%

% Please keep new commands to a minimum, and use \newcommand not \def to avoid
% overwriting existing commands. Example:
%\newcommand{\pcm}{\,cm$^{-2}$}	% per cm-squared
\setlength{\parindent}{0pt}
%%%%%%%%%%%%%%%%%%%%%%%%%%%%%%%%%%%%%%%%%%%%%%%%%%
\newcommand{\Msun}{\,{\rm M}$_{\odot}$\,}
\newcommand{\Msunh}{\,{\rm M}$_{\odot}$\,\ifmmode h^{-1}\else $h^{-1}$\fi}
\newcommand{\Mpch}{\,{\rm Mpc}\,\ifmmode h^{-1}\else $h^{-1}$\fi}
\newcommand{\kpch}{\,{\rm kpc}\,\ifmmode h^{-1}\else $h^{-1}$\fi}
\newcommand{\kpc}{\,{\rm kpc}\,}
%%%%%%%%%%%%%%%%%%% TITLE PAGE %%%%%%%%%%%%%%%%%%%

% Title of the paper, and the short title which is used in the headers.
% Keep the title short and informative.
\title[Galaxy Assembly Bias]{The Assembly Time Dependence of Galaxy Clustering}

% The list of authors, and the short list which is used in the headers.
% If you need two or more lines of authors, add an extra line using \newauthor
\author[Camargo, Y.BUSCAR ADS \& Forero-Romero]{
Y. Camargo,$^{1}$\thanks{E-mail: mn@ras.org.uk}
J. E. Forero-Romero,$^{2}$\thanks{E-mail: je.forero@uniandes.edu.co}
\\
% List of institutions
$^{1}$Universidad Nacional de Colombia, Bogot\'a, Colombia\\
$^{2}$Departamento de F\'isica, Universidad de los Andes, Cra. 1 No.
18A-10, Edificio Ip, Bogot\'a, Colombia\\
}

% These dates will be filled out by the publisher
\date{Accepted XXX. Received YYY; in original form ZZZ}

% Enter the current year, for the copyright statements etc.
\pubyear{2015}

% Don't change these lines
\begin{document}
\label{firstpage}
\pagerange{\pageref{firstpage}--\pageref{lastpage}}
\maketitle

% Abstract of the paper
\begin{abstract}
    We use a high resolution cosmological hydro-dynamical simulation
    to investigate stellar mass assembly 
    and its impact on clustering at $z=0$.
    Using the merger trees for galaxies in the range $10^{9}$\Msunh $\leq M_{\star} \leq 10^{12.5}$ \Msunh we first find a transition mass of $10^{10.5}$\Msunh that divides two
    correlation regimes between assembly time and stellar mass.
    Above that mass, late assembly times correlate with more massive
    galaxies; below that mass late assembly correlates with less massive
galaxies.
We then quantify how, at fixed stellar mass, galaxies with an early
assembly
are more clustered than late assembly galaxies, regardless if the galaxy
is a central or a satellite.
Finally, for stellar masses below $10^{10}$ \Msunh we find 
cuts in the star-formation rate and the $(g-r)$ colour that mimic
the bias found for early/late assemblying galaxies.
These results can be used to.
Furthermore, these findings give further support to the idea that at
$z=0$  massive galaxies above $10^{10.5}$\Msunh do not show a strong
stellar mass assembly bias.
\end{abstract}

% Select between one and six entries from the list of approved keywords.
% Don't make up new ones.
\begin{keywords}
keyword1 -- keyword2 -- keyword3
\end{keywords}

%%%%%%%%%%%%%%%%%%%%%%%%%%%%%%%%%%%%%%%%%%%%%%%%%%

%%%%%%%%%%%%%%%%% BODY OF PAPER %%%%%%%%%%%%%%%%%%

\section{Introduction}
In the Standard Cosmological Model ($\Lambda$CDM) the galaxies grow inside dark matter halos product of gravitational instabilities that evolve due to successive merges, in this scenario, dark matter halos form first, and the physical properties of galaxies are determined by cooling and condensation of gas within of halos, so the formation and evolution of galaxies is linked to its halo. 
Traditionally is assumed that clustering of halos dependent only its halo mass, i.e., more massive halos being more strongly clustered than less massive halos \citep{1984ApJ...284L...9K}??, but \citet{2005MNRAS.363L..66G} showed that at fixed halo mass, oldest halos tend to be located in anisotropy environments while its youngest counterparts in less dense environments known as assembly halo bias, this effect becomes very large for the lowest masses. 
In terms the galaxy-halo relationship the Halo Occupation Distribution model (HOD) is used to explain the observed effects of different types of galaxies towards the conditional distribution function $P(N_G|M_H)$, this statistical connection ignores the possibility that galaxy properties may be correlated with halo properties different to halo mass, i.e., this model assumes that halo mass alone suffices to determine the physical properties of halo inner galaxy population, however, recent studies have been shown additional dependencies to time halo formation such as spin, shape, velocity dispersion and concentration, commonly referred as secondary halo assembly bias, in terms of galaxy-halo relationship, the dependence on galaxy clustering on the secondary assembly halo bias is yet subject of discussion, to this relationship usually refereed as galaxy assembly bias.\\

From the observational point of view, \citet{Lacerna_2014} find in SSDS a fixed halo mass $M_h\approx 10^{11.8}h^{-1}M_\odot$ central galaxies show a weak but significant dependence of clustering amplitude with the age, i.e, old central galaxies have a higher clustering that young central galaxies also found in its mock catalog, similarly \citet{2016PhRvL.116d1301M} show present significant evidence of halo assembly bias for SDSS based on weak lensing signals to two samples of very similar halo mass of $M_{200}\approx 1.9\times 10^{14}h^{-1}M_\odot$, on the other hand, \citet{2019MNRAS.485.1196Z} and \citet{2016ApJ...819..119L} do not find convincing evidence of assembly bias in SDSS. In the context of the galaxy quenching the models predict a strong correlation between a fraction quenched and large scale density environment, using central galaxies $M_{\star} \gtrsim 10^{10}M_\odot h^{-2}$ in SDSS \citet{2017MNRAS.472.2504T} show that the halo formation history has a small but statistically significant impact on quenching of star formation at high masses, while the process in low-mass central galaxies is uncorrelated with halo formation history, \citet{2016MNRAS.457.4360Z} using clustering and galaxy-galaxy lensing of red and blue galaxies in SDSS find that the halo quenching model provides more fitting to the galaxies above $10^{11}M_\odot h^{-2}$, predicting the average halo mass of red and blue central galaxies although it has also been shown that quenching process has a limited correlation with halo formation history \citep{2018MNRAS.478.4487T}. Using hydrodynamics simulations and N-body simulations the galaxy assembly bias effect is more visible, in terms of variations galaxy occupancy of the dark matter halos, halos at low mass ($\leq 10^{10}$\Msunh) in most dense environment and early formed halos expected host a central galaxy and fewer satellite galaxies that those in the least dense environments and late-formed halos \citep{2018MNRAS.480.3978A} in the galactic conformity effect galaxies with stellar mass $M_\star > 2\times 10^9 M_\odot$ shows a significant signal out to 10 Mpc \citep{2016MNRAS.455..185B}, moreover in Illustris simulations has been demonstrated a stronger correlation between the central galaxies and the peak maximum circular velocity of their hosting haloes \citep{2018arXiv181211210X}\citep{2005ApJ...630....1Z}.
Also, Semi Analytic models (SAM) show contradictory results, some studies find that the clustering of galaxies depend not only halo mass but also on the secondary halo properties, halo central galaxies are differently affected by assembly bias than are galaxies of all types (see, e.g., \citep{2019MNRAS.486..582P}; \citep{2007MNRAS.374.1303C};\citep{2019MNRAS.484.1133C}; \citep{2014ApJ...794...74J}), while authors as \citet{2014MNRAS.443.3044Z} conclude that the galaxy-halo relationship inferred from galaxy clustering is subject to significant systematic errors induced by assembly bias. If the galaxy properties depend on the secondary properties of its parent halo it remains unclear especially on the observational point of view.\\


%The galaxy is a primer tracer of galaxy formation and galaxy clustering is the most basic test to cosmological models.

As previously we mentioned the investigations about the age-clustering dependence have mainly focused on halo population detecting this effect (assembly halo bias), i.e., oldest halos are more clustered than younger haloes, however, the galaxies are the tracers of formation structure and are used to constrain the cosmological parameters. Some authors focus their studies on age-clustering relation in galactic populations, e.g., \citet{2012A&A...539A..46A} study the properties of galaxy pairs in high-density environments in SSDS-DR7 detecting young groups and young clusters are associated with low-density global environments, \citet{2007MNRAS.378..777R} analyzing the relation between age and clustering within halo catalogs found that galaxies of a given velocity circular will have formed earlier if they lie in groups or clusters today, the other hand, using semi-analytic galaxy formation models and observational data at fixed stellar mass \citet{2013MNRAS.433..515W} predict that the clustering of central galaxies depends on the specific star formation rate, found more passive galaxies have a higher clustering amplitude, but \citet{Lacerna_2014} also find a weak but significant dependence of clustering amplitude whit the age of central galaxies in SDDS, at fixed halo mass they found that old galaxies show a higher clustering relative the young population and \citet{Berlind:2006eb} found a connection between halo age and central galaxy, halos that assemble earlier likely contain redder central galaxies than recently assembled halos of the same mass. The galaxy assembly bias study in this paper is therefore the relation between the galaxy age dependence and galaxy clustering. In this letter, we study the dependence of galaxy clustering on galaxy assembly for central and satellite galaxies using the cosmological hydro-dynamical simulation Illustris TNG300-1. In this way, the definition for galaxy assembly bias indicates the dependence of galaxy clustering on a secondary galactic property as the assembly time.\\

The outline of this letter is as follows, in section \ref{sec:simul} we describe the IllustrisTNG300-1 simulations. In section \ref{sec:galactic_prop} we detail the construction of galaxy sample in IllustrisTNG300-1, the stellar mass range, the assembly times and the definition of our age dependence of clustering for central and satellite galaxies through the relative bias. Finally, the results and conclusions are discussed in section \ref{sec:conclu}.

\newpage

\section{Simulations}
\label{sec:simul} % used for referring to this section from elsewhere

In this letter we use IllustrisTNG300-1\footnote{The Next Generation Illustris} magnetohydrodynamical simulations of galaxy formation performed with AREPO code \citep{2018MNRAS.473.4077P}, the simulation box sizes employed is a cubic box of side $205$\Mpch which allows a high level of resolution, this enables the study of galaxy clustering, the analysis of rare objects such as galaxy clusters and provides the largest galaxy sample\footnote{tng-project.org}.
TNG300-1 follows the evolution of $2500^3$ DM particles of mass $3.98 \times 10^7$\Msunh. IllustrisTNG simulations follow the evolution of dark matter from early time  ($z=20.05$) to presente ($z=2.22\times 10^{-16}$) assuming $\Lambda CDM$ cosmology with the parameters $\Omega_m=0.38089$, $\Omega_b=0.0486$, $\Omega_\Lambda= 0.6911$ and $h=0.6774$ consistent with Planck2015 \citep{2016A&A...594A..13P}.\\
%IllustrisTNG300-1 simulation includes a model for galaxy formation that includes the stellar formation and evolution, chemical enrichment

\begin{figure}
    \centering
    \includegraphics[width=1\columnwidth]{figuras/Histogramas.pdf}
    \caption{Stellar mass functions for all galaxies and the partition into centrals and satellites. We focus our study in the stellar mass range presented in this plot.}
    \label{fig:stellar_fuction}
\end{figure}

\begin{figure}
    \centering
     \includegraphics[width=1\columnwidth]{figuras/Number_Galaxies.pdf}
    \caption{Number of galaxies as a function of their host halo mass.}
    \label{fig:abundance}
\end{figure}

\begin{figure*}
    \centering
     \includegraphics[width=1\columnwidth]{figuras/CH.pdf}
    \includegraphics[width=0.9\columnwidth]{figuras/SH.pdf}
    \caption{Relationship between stellar mass and the parent dark matter halo mass.}
    \label{fig:stellar_to_halo}
\end{figure*}

\begin{figure}
    \centering
    \includegraphics[width=1\columnwidth]{figuras/median_assembly.pdf}
    \caption{Median redshift of assembly as a function of stellar mass.
    Different symbols correspond to satellites, centrals or all galaxies.
    The stellar mass around $10^{10.5}$\Msunh shows a transition between two 
    regimes of \emph{upsizing} and \emph{downsizing}.}
    \label{fig:median_assembly}
\end{figure}
 
 \begin{figure*}
    \centering
    \includegraphics[width=1.8\columnwidth]{figuras/scatter_assembly.pdf}
    \caption{Comparison of the spatial distribution of \emph{early} and \emph{late} assembling galaxies.
    The galaxies included in the plot have stellar masses around $10^{10}$\Msunh. 
    Each panel projects all the galaxy positions over the $x$-$y$ plane. 
    The split between \emph{late} and \emph{early} corresponds to the first and fourth quartile in the redshift assembly distribution for all galaxies with the same mass.
    Each panel has $\sim7$k galaxies. }
    \label{fig:comparison}
\end{figure*}

%%%Galaxy properties

\section{Galaxy Properties}
\label{sec:galactic_prop}
In IllustrisTNG simulations, the DM halos are identified using the Friends-of-Friends (FoF) algorithm with a linking length of 0.2 times mean interparticle separation. The substructure (subhalos) are identified with the SUBFIND algorithm modified, this consist in calculated the density field for all particles and gas cells using an adaptive smoothing length corresponding to the distribution of DM particles, the subhalos candidates are defined determining the first isodensity contour that passes through a saddle point of the density field then the subhalo candidate is subject to a gravitational unbinding procedure but during this process, the gas thermal energy is also taken into account \citep{2015MNRAS.449...49R}. \\

To define the simulated galaxies we use the measurement mass within twice the stellar half mass radius of each subhalo \citep{2018MNRAS.475..676S} with $M_\star >0$ in TNG300-1 catalog groups in $z=0$, we discarded subhalos with no tree that contain galaxies with a stellar mass in the range $4.05\times 10^{6}$\Msunh to $2.06\times 10^{10}$\Msunh, in this data set the objects with stellar mass $M_\star>10^9$\Msunh was 3827. The central subhalos were identified as the ID into the Subhalo table as of the most massive Subfind group within this FoF group, these central subhalos may contain one or more satellite subhalos. Finally, the catalog to central galaxies is 720797 and 484732 to satellite galaxies. We selected centrals and satellites galaxies in a range of stellar mass between $10^{9}$\Msunh $\leq M_{\star} \leq 10^{12.5}$ \Msunh, thus our data set of central galaxies contain 128258 and 89725 of satellite galaxies.\\

Figure \ref{fig:stellar_fuction} shows the stellar mass functions for central galaxies, satellite galaxies and all galaxy sample in $z=0$, the stellar mass selected are $10^9$\Msunh $\leq M_\star \leq 10^{12.5} $\Msunh.\\

In fig \ref{fig:abundance} display the occupancy variations for central and satellite galaxies. Initially, halos with low halo mass start hosting more central galaxies than satellites. In the case of central galaxies the number of objects within halos rapidly increases with the halo mass, but to halos with $M_h \gtrsim 10^{10.5}$ \Msunh the number of galaxies central galaxies decreases. In the case of satellite galaxies, the number of objects inside halos also tend to increase with the halo mass, however to halos with $M_h \gtrsim 10^{11.5}$ \Msunh the number of this galaxies inside tends to remain constant, either case the number of galaxies does not scale linearly with the halo mass, in concordance to what found by \citet{2019ApJ...887...17Z} but in terms of concentration. For the halo mass in the fig \ref{fig:stellar_to_halo}, shows clearly that halos more massive content more massive central galaxies. The central and satellite galaxies evolve in different forms inside halos of the same mass, therefore relating to halo mass with the galaxy content exists a bias.\\
%\textbf{we show the relationship between the stellar mass as a function of dark matter halo mass and in color-coding the stellar assembly time}

%%%% Assembly times
\subsection{Assembly times}
\label{sec:assembly_times}
To estimate the formation assembly time for central and satellite galaxies, we use the merger trees\footnote{baryonic merger trees} constructed using SUBLINK algorithm at the subhalo level available in IllustrinTNG database, the descendant candidates are identified for each subhalo as those subhalos in the next snapshot that have common particles with the subhalo in question, the firs progenitor is defined as the one 'most massive history' behind it \citep{2015MNRAS.449...49R}, the formation assembly time is identified as the redshift assembly $z_{for}$ when the galaxies has exactly half stellar mass of its final stellar mass. We selected galaxies with a mass range of $10^{9}$\Msunh $\leq M_{\star} \leq 10^{12.5}$ \Msunh as a result, the final data set contain 128258 to central galaxies and 89725 to satellite galaxies. We referred galaxies are "youngest" as recent galaxy formation assembly, i.e., galaxies into the first quartile in each galaxy sub-sample and "oldest" galaxies as early galaxy formation assembly, i.e., in the last quartile. As expected, the satellite galaxies tend to be older than central ones because the satellite subhalos may survive as bound entities inside a bigger halo.\\

In fig \ref{fig:comparison}, we provide a visual impression of galaxy distribution of early assembly galaxies (older galaxies) and late assembly galaxies (younger galaxies). It is clear that galaxies that early assembly tends to be located in clusters to be more tightly clustered than its late-assembly counterparts. 

%%%%%Galaxy bias
\subsection{Galaxy Bias}
To quantify the galaxy clustering, we estimate the relative galaxy bias by taking the ratio of the 2-point correlation function of two samples:
%
\begin{equation}
b_r(r, S|A)= \frac{\xi_S(r)}{\xi_A(r)}, 
\label{eq:relative}
\end{equation}
%
where $\xi_A$ corresponds to the correlation function from a general sample $A$ that includes all the galaxies in a fixed stellar mass bin, while $\xi_S$ is the correlation function computed for a sub-sample, $S$, of galaxies in $A$. In this paper the sub-samples are built by taking the first and last quartile in the distributions of redshift assembly.\\

In fig \ref{fig:galaxy_bias}, we present the relative galaxy bias as a function of stellar mass for central, satellites, and all galaxies separately. The relative bias for early assembly galaxies (older galaxies) is higher in comparison with the later assembly galaxies (younger galaxies) however, as the mass increases for older galaxies the relative bias tends to decreases, on the other hand, for the younger population of galaxies the relative bias tend to stay almost constant especially for all and central galaxies, for satellite galaxies this trend continues up to transitional mass for both older and younger populations.

%%%Conclusions
\section{Discussion and conclusions}
\label{sec:conclu}
In this letter, we used the cosmological hydro-dynamical simulation IllustrisTNG300-1 to study how the clustering of galaxies depends and formation galaxy time (assembly stellar) for central and satellite galaxies with a stellar mass in the range $10^{9}$\Msunh $\leq M_{\star} \leq 10^{12.5}$.
%In this letter, we used the cosmological hydro-dynamical simulation IllustrisTNG300-1 to study the galaxy clustering dependence on assembly stellar mass in galaxies with a stellar mass in the range $10^{9}$\Msunh $\leq M_{\star} \leq 10^{12.5}$, investigated the impact on clustering of galaxies in terms of the assembly times and quantify the clustering bias towards to relative bias. We have shown a clustering-age (assembly stellar mass) dependence for central and satellite galaxies similar to found by \citet{2005MNRAS.363L..66G} for halos.\\
\begin{figure*}
    \centering
    \includegraphics[width=2.0\columnwidth]{figuras/bias_galaxies.pdf}
    \caption{Relative galaxy bias as a function of stellar mass.
    We consider separately all galaxies, satellites and centrals.
    The bias is higher for early assembling galaxies.}
    \label{fig:galaxy_bias}
\end{figure*}

In terms of the number of galaxies inside halos, the fig \ref{fig:abundance} show the number of galaxies as a function of halo mass, in this figure is evident that there is a strong dependence between the halo mass and the galaxy content especially to central galaxies, in this case to halos with $M_h \gtrsim 10^{10.5}$ \Msunh the number of central galaxies decrease, low mass halos start hosting more central galaxies than satellites, however as the halo mass increases above this mass, this trend is reversed. In the case of the number of satellites galaxies, for halos with $M_h\gtrsim 10^{11.5}$ \Msunh, the number of satellite galaxies gives the impression to remains constant. %We also found in fig \ref{fig:stellar_to_halo} that central galaxies inside halos of mass $10^{12}$\Msunh $\lesssim M_h \lesssim 10^{13}$\Msunh and stellar mass to $10^{10}$\Msunh $\lesssim M_\star \lesssim 10^{10.5}$\Msunh tend to early assembly, however to satellite galaxies is not clear the dependence on the time of stellar assembly and the mass of its halos parent. 


In fig \ref{fig:median_assembly}, we explore the stellar mass dependence with the median assembly times (galaxy formation time) for satellite and central galaxies in more detail. In this figure, we can see that while the stellar mass increases the variation of median $z$ assembly stellar mass growth smooth (we dubbed this trend upsizing) to "transitional stellar mass" in $M_\star 10^{10.5}$ \Msunh, showing two regimens of upsizing and downsizing for central and satellite galaxies in the upsizing regimen, the median assembly stellar mass increases with the stellar mass whilst the downsizing regimen this trend is inverted for both central and satellites galaxies, less massive galaxies are the ones with a late assembly. %In concordance, the fig \ref{fig:stellar_to_halo} show that central galaxies with a stellar in the range $10^{10}$ \Msunh $\leq  M_\star \leq $ $10^{10.5}$\Msunh are earlier assembled, while for satellite galaxies this relationship is not very clear.\\


In fig \ref{fig:comparison}, we present the distribution of early assembly stellar mass (older galaxies) and late assembly stellar mass (younger galaxies) for a given stellar mass, revealing that early-assembly galaxies tend to be located in overdense regions than its late-assembly counterparts. 
Using the two-point correlation function, we quantify the clustering towards the relative bias defined as the equation \ref{eq:relative}. In fig \ref{fig:galaxy_bias}, we present the relative bias as a function of stellar mass for all, satellites, and central galaxies separately. At fixed stellar mass older galaxies to low stellar mass tend to be more tightly clustered than its late-assembly counterpart according to e.g., \citep{Lacerna_2014}, \citep{2013MNRAS.433..515W}, \citep{2009MNRAS.394.2229Z}. We find that in all cases the relative bias is higher for older galaxies and this trend is never reversed although stronger for low stellar mass, however as the mass increases the relative bias tends to decreases up to transitional mass and the clustering strength is very similar for both older and younger satellite galaxies.
By the other hand, in the central galaxies, the relative bias decrease up to stellar mass in range to $10^{10.75}$\Msunh $\lesssim M_\star \lesssim$ $10^{11}$\Msunh but is not possible in this range of stellar mass make a distinction between younger and older galaxies because the strength of clustering is very similar. 
For the younger galaxies, the relative bias remains constant showing a slight increase in the stellar masses mentioned above, for galaxies above this stellar mass the relative bias increases to older galaxies, contrary to occur for satellite galaxies, the relative bias increases but immediately decreases for both younger and older populations. For all galaxies, the previous trend continues but the relative bias decreases up to $M_\star = 10^{11}$ \Msunh in this case always is possible to distinguish older to younger galaxies populations.

Finally, we found clear evidence for assembly bias on central/satellite galaxies in favor of $M_star \leq 10^{10.5}$\Msunh, as we show in fig \ref{fig:galaxy_bias} and fig \ref{fig:median_assembly}, the younger galaxies tend to be in regions that look almost uniform while the older galaxies tend to habit in overdense regions. We find a transitional stellar-mass indicating especially in central galaxies not is possible to distinguish between older or younger galaxies. 
To conclude worth noting, that the central and satellite galaxies evolve in different ways, however, the relationship age-clustering is the same, i.e, older galaxies prefer overdense regions than younger galaxies so that the galaxy properties should depend significantly on the assembly history of their haloes, the old haloes follow the large-scale cosmic web quite closely, while the distribution of young haloes looks almost uniform.
%%%%%%%%%%%%%%%%%%%%%%%

%Fig. \ref{fig:figure3} shows that while the  mass increases the bias factor also grows (it is not linear), but for a lowest mass galaxies the bias factor is small for young galaxies than for the older ones. 
%However, as the mass increases, this difference in the bias factor become smaller and for a larger mass can not be distinguished the time of formation  of the galaxy by the bias. This is not only seen for the Illustris TNG300-1 simulation this fact is repeated in all simulations as we can seen in fig. \ref{fig:figure1}. In fig. \ref{fig:figure5} it is clearly seen 

 %Since it is plausible that galaxy properties should depend significantly on the assembly history of their haloes,
%the old haloes follow the large-scale cosmic web quite closely, while the distribution of young haloes looks almost uniform.
%the low mass structures are dominated by dark matter, while the larger mass structures are not expected to have high dark matter content.





\section*{Acknowledgements}

a gustavo por prestarnos el aparato jijij


\bibliographystyle{mnras}
\bibliography{bibliography.bib}


% Don't change these lines
\bsp	% typesetting comment
\label{lastpage}
\end{document}